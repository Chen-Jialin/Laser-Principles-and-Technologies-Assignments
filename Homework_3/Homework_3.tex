% !TEX program = pdflatex
% Laser Principles & Technologies Homework_3
\documentclass[12pt,a4paper]{article}
\usepackage[margin=1in]{geometry} 
\usepackage{amsmath,amsthm,amssymb,amsfonts,enumitem,fancyhdr,color,comment,graphicx,environ}
\pagestyle{fancy}
\setlength{\headheight}{65pt}
\newenvironment{problem}[2][Problem]{\begin{trivlist}
\item[\hskip \labelsep {\bfseries #1}\hskip \labelsep {\bfseries #2.}]}{\end{trivlist}}
\newenvironment{sol}
    {\emph{Solution:}
    }
    {
    \qed
    }
\specialcomment{com}{ \color{blue} \textbf{Comment:} }{\color{black}}
\NewEnviron{probscore}{\marginpar{ \color{blue} \tiny Problem Score: \BODY \color{black} }}
\usepackage[UTF8]{ctex}
\lhead{Name: 陈稼霖\\ StudentID: 45875852}
\rhead{PHYS1553 \\ Laser Principles \& Technologies \\ Semester Fall 2019 \\ Assignment 3}
\begin{document}
\begin{problem}{3.1}
腔长为$0.5$m的氩离子激光器,发射中心频率为$\nu_0=5.85\times10^{14}$Hz,荧光线宽$\Delta\nu=6\times10^8$Hz。问它可能存在几个纵模,相应的$q$值为多少?(设$\mu=1$)
\end{problem}
\begin{sol}
满足谐振条件的相邻纵模频率之差为
\begin{equation}
\Delta\nu_q=\frac{c}{2\mu L}=\frac{3\times10^8\text{m}/\text{s}}{2\times1\times0.5\text{m}}=3\times10^8\text{Hz}
\end{equation}
发射中心频率对应的纵模序数为
\begin{equation}
q_0=\frac{\nu_0}{\Delta\nu_q}=\frac{5.58\times10^{14}\text{Hz}}{3\times10^8\text{Hz}}=1950000
\end{equation}
荧光线宽范围内可能存在的纵模数为
\[
n=\frac{\Delta\nu}{\Delta\nu_q}+1=\frac{6\times10^8\text{Hz}}{3\times10^8\text{Hz}}+1=3
\]
相应的纵模序数为
\[
q_{-1}=q_0-1=1949999,\quad q_0=1950000,\quad q_{+1}=q_0+1=1950001
\]
\end{sol}

\begin{problem}{3.7}
一个共焦腔(对称)的$L=0.40$m,$\lambda=0.6328\mu$m,求束腰半径和离腰处$56$cm处的光束有效截面半径。
\end{problem}
\begin{sol}
束腰半径为
\begin{equation}
w_0=\sqrt{\frac{\lambda L}{2\pi}}=\sqrt{\frac{0.6328\times10^{-6}\text{m}\times0.40\text{m}}{2\pi}}=2.0\times10^{-4}\text{m}=0.20\text{mm}
\end{equation}
离腰$56$cm处的光束有效截面半径为
\begin{align}
\nonumber w(z=56\text{cm})=&w_0\sqrt{1+\left(\frac{\lambda z}{\pi w_0^2}\right)^2}\\
\nonumber=&2.0\times10^{-4}\text{m}\sqrt{1+\left(\frac{0.6328\times10^{-6}\text{m}\times56\times10^{-2}m}{\pi(2.0\times10^{-4}\text{m})^2}\right)^2}\\
=&6.0\times10^{-3}\text{m}=6.0\text{mm}
\end{align}
\end{sol}

\begin{problem}{3.11}
试从式(3-88)出发,证明用最佳透射率表示的非均匀增宽激光器的最佳输出功率为
\[
P_m=AI_s\frac{t_m^2}{(a-t_m)}
\]
\end{problem}
\begin{sol}
根据式(3-88),纵模$\nu$的输出频率为
\begin{equation}
P(\nu)=\frac{1}{2}At_1I_s\left[\left(\frac{2LG_D^0(\nu)}{a_1+t_1}\right)^2-1\right]
\end{equation}
上式关于透射率求导得
\begin{equation}
\frac{dP}{dt_1}=\frac{1}{2}AI_s\left[\frac{(2LG_D^0(\nu))^2(a_1-t_1)}{(a_1+t_1)^3}-1\right]
\end{equation}
当达到最佳输出时,
\begin{equation}
\left.\frac{dP}{dt_1}\right|_{t_1=t_m}=0\Longrightarrow\left(2LG_D^0(\nu)\right)^2=\frac{(a_1+t_m)^3}{(a_1-t_m)}
\end{equation}
代入式(3-88)中得到
\begin{equation}
P_m=\frac{1}{2}AI_st\left[\frac{\frac{(a_1+t_1)^3}{(a_1-t_1)}}{(a_1+t_1)^2}-1\right]=AI_s\frac{t_m^2}{(a_1-t_m)}
\end{equation}
\end{sol}
\end{document}