% !TEX program = pdflatex
% Laser Principles & Technologies Homework_1
\documentclass[12pt,a4paper]{article}
\usepackage[margin=1in]{geometry} 
\usepackage{amsmath,amsthm,amssymb,amsfonts,enumitem,fancyhdr,color,comment,graphicx,environ}
\pagestyle{fancy}
\setlength{\headheight}{65pt}
\newenvironment{problem}[2][Problem]{\begin{trivlist}
\item[\hskip \labelsep {\bfseries #1}\hskip \labelsep {\bfseries #2.}]}{\end{trivlist}}
\newenvironment{sol}
    {\emph{Solution:}
    }
    {
    \qed
    }
\specialcomment{com}{ \color{blue} \textbf{Comment:} }{\color{black}} %for instructor comments while grading
\NewEnviron{probscore}{\marginpar{ \color{blue} \tiny Problem Score: \BODY \color{black} }}
\usepackage[UTF8]{ctex}
\lhead{Name: 陈稼霖\\ StudentID: 45875852}
\rhead{PHYS1553 \\ Laser Principle \& Technology \\ Semester Fall 2019 \\ Assignment 1}
\begin{document}
\begin{problem}{1.3}
已知氢原子第一激发态($E_2$)与基态($E_1$)之间能量差为$1.64\times10^{-18}$J,火焰($T=2700K$)中含有$10^{20}$个氢原子。设原子按玻尔兹曼分布,且$4g_1=g_2$。求\\
(1)能级$E_2$上的原子数$n_2$为多少?\\
(2)设火焰中每秒发射的光子数为$10^8n_2$,求光的功率为多少瓦?
\end{problem}
\begin{sol}
\\(1)总原子数
\begin{equation}
n_1+n_2=10^{20}
\end{equation}
根据玻尔兹曼分布
\begin{equation}
\frac{n_2/g_2}{n_1/g_1}=\frac{n_2}{4n_1}=e^{-\frac{(E_2-E_1)}{kT}}=7.67\times10^{-20}
\end{equation}
以上两式联立得能级$E_2$上的原子数
\begin{equation}
n_2=31
\end{equation}
(2)光的功率
\begin{gather}
E=10^8n_2(E_2-E_1)=5.1\times10^{-9}W
\end{gather}
\end{sol}

\begin{problem}{1.11}
静止氖原子的3S$_2\rightarrow$2P$_4$,谱线的中心波长为$0.6328\mu$m,设氖原子分别以$\pm0.1c$,$\pm0.5c$的速度向着接收器运动,问接收到的频率各为多少?
\end{problem}
\begin{sol}
静止氖原子的3S$_2\rightarrow$2P$_4$的发光频率为
\begin{equation}
\nu=\frac{c}{\lambda}=4.7408\times10^{14}\text{Hz}
\end{equation}
根据多普勒效应,当氖原子以$0.1c$向着接收器运动,接收到的频率
\begin{equation}
\nu_{0.1c}=\sqrt{\frac{1+0.1c/c}{1-0.1c/c}}\nu=5.2\times10^{14}\text{Hz}
\end{equation}
当氖原子以$-0.1c$向着接收器运动,接收到的频率
\begin{equation}
\nu_{-0.1c}=\sqrt{\frac{1-0.1c/c}{1+0.1c/c}}\nu=4.3\times10^{14}\text{Hz}
\end{equation}
当氖原子以$0.5c$向着接收器运动,接收到的频率
\begin{equation}
\nu_{0.5c}=\sqrt{\frac{1+0.5c/c}{1-0.5c/c}}\nu=8.2\times10^{14}\text{Hz}
\end{equation}
当氖原子以$-0.5c$向着接收器运动,接收到的频率
\begin{equation}
\nu_{-0.5c}=\sqrt{\frac{1-0.5c/c}{1+0.5c/c}}\nu=2.7\times10^{14}\text{Hz}
\end{equation}
\end{sol}

\begin{problem}{1.13}
(1)一质地均匀的材料对光的吸收为$0.01\text{mm}^{-1}$,光通过$10$cm长的该材料后,出射光强为入射光强的百分之几?\\
(2)一束光通过长度为$1$m的均匀激活的工作物质,如果出射光强是入射光强的两倍,试求该物质的增益系数。
\end{problem}
\begin{sol}
\\(1)
\begin{equation}
\frac{I}{I_0}\times100\%=e^{-Az}\times100\%=36.8\%
\end{equation}
故出射光强为入射光强的$36.8\%$。\\
(2)该物质的增益系数
\begin{gather}
\frac{I}{I_0}=e^{Gz}=2\Longrightarrow G=\ln2m^{-1}=0.693m^{-1}
\end{gather}
\end{sol}
\end{document}